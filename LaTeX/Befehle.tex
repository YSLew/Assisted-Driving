% Eigene Befehle und typographische Auszeichnungen f�r diese

% einfaches Wechseln der Schrift, z.B.: \changefont{cmss}{sbc}{n}
\newcommand{\changefont}[3]{\fontfamily{#1} \fontseries{#2} \fontshape{#3} \selectfont}

% Abk�rzungen mit korrektem Leerraum 
\newcommand{\ua}{\mbox{u.\,a.\ }}
\newcommand{\zB}{\mbox{z.\,B.\ }}
\newcommand{\dahe}{\mbox{d.\,h.\ }}
\newcommand{\Vgl}{Vgl.\ }
\newcommand{\bzw}{bzw.\ }
\newcommand{\evtl}{evtl.\ }

\newcommand{\abbildung}[1]{Abbildung~\ref{fig:#1}}

\newcommand{\bs}{$\backslash$}

% erzeugt ein Listenelement mit fetter �berschrift 
\newcommand{\itemd}[2]{\item{\textbf{#1}}\\{#2}}

% einige Befehle zum Zitieren --------------------------------------------------
\newcommand{\Zitat}[2][\empty]{\ifthenelse{\equal{#1}{\empty}}{\citep{#2}}{\citep[#1]{#2}}}

% zum Ausgeben von Autoren
\newcommand{\AutorName}[1]{\textsc{#1}}
\newcommand{\Autor}[1]{\AutorName{\citeauthor{#1}}}

% verschiedene Befehle um W�rter semantisch auszuzeichnen ----------------------
\newcommand{\NeuerBegriff}[1]{\textbf{#1}}
\newcommand{\Fachbegriff}[1]{\textit{#1}}

\newcommand{\Eingabe}[1]{\texttt{#1}}
\newcommand{\Code}[1]{\texttt{#1}}
\newcommand{\Datei}[1]{\texttt{#1}}

\newcommand{\Datentyp}[1]{\textsf{#1}}
\newcommand{\XMLElement}[1]{\textsf{#1}}
\newcommand{\Webservice}[1]{\textsf{#1}}

% f�r Formelverzeichnis --------------------------------------------------------------------
\DeclareNewTOC[% 
  indent=0pt,% kein Einzug im Verzeichnis 
  hang=2em,% Einzug f�r den Text im Verzeichnis 
  type=equation 
]{loe} 

\AtBeginDocument{% 
  \newcaptionname{ngerman}\equationname{Formel}% 
  \newcaptionname{ngerman}\listequationname{Formelverzeichnis}% 
} 

% f�r Bildquellen ----------------------------------------------------------------------------
\newcommand*{\quelle}{
	\footnotesize Quelle:
}

% Daf�r sorgen, dass Eintr�ge immer dann erzeugt werden k�nnen, wenn auch Tags 
% gesetzt werden: 
\makeatletter 
\newcommand*{\@currententry}{} 
% Zwei amsmath-Anweisungen �ndern: 
\g@addto@macro\make@display@tag{\set@currententry}% 
\def\tagform@#1{\maketag@@@{(\ignorespaces#1\unskip\@@italiccorr)}% 
  \set@currententry} 
\newcommand*{\set@currententry}{% 
  \typeout{set current entry}% 
  \ifx\@currententry\@empty\else 
    \addcontentsline{loe}{equation}{\protect\numberline{\@currentlabel}% 
      \@currententry}% 
    \global\let\@currententry\@empty 
  \fi 
} 

% Neue Benutzeranweisung 
\newcommand*{\equationentry}[1]{% 
  \gdef\@currententry{#1}% 
} 
\makeatother
% ------------------------------------------------------------------------------------------------