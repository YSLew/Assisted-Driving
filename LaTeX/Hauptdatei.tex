% ------------------------------------------------------------------------------
% Formatvorlage f�r Masterarbeiten an der Ohm-Hochschule N�rnberg
% ------------------------------------------------------------------------------
%   erstellt von Stefan Macke, 24.04.2009
%   http://blog.stefan-macke.de

% Dokumentenkopf ---------------------------------------------------------------
%   Diese Vorlage basiert auf "scrreprt" aus dem koma-script.
% ------------------------------------------------------------------------------
\documentclass[
    11pt, % Schriftgr��e
    DIV10,
    ngerman, % f�r Umlaute, Silbentrennung etc.
    a4paper, % Papierformat
    oneside, % einseitiges Dokument
    titlepage, % es wird eine Titelseite verwendet
    parskip=half, % Abstand zwischen Abs�tzen (halbe Zeile)
    headings=normal, % Gr��e der �berschriften verkleinern
    pointlessnumbers, %keine Punkte nach Kapitelnummer
    listof=totoc, % Verzeichnisse im Inhaltsverzeichnis auff�hren
    bibliography=totoc, % Literaturverzeichnis im Inhaltsverzeichnis auff�hren
    index=totoc, % Index im Inhaltsverzeichnis auff��hren
    captions=tableheading, % Beschriftung von Tabellen unterhalb ausgeben
    final % Status des Dokuments (final/draft)
]{scrreprt}

% Meta-Informationen -----------------------------------------------------------
%   Informationen �ber das Dokument, wie z.B. Titel, Autor, Matrikelnr. etc
%   werden in der Datei Meta.tex definiert und k�nnen danach global
%   verwendet werden.
% ------------------------------------------------------------------------------
% Meta-Informationen -----------------------------------------------------------
%   Definition von globalen Parametern, die im gesamten Dokument verwendet
%   werden k�nnen (z.B auf dem Deckblatt etc.).
%
%   ACHTUNG: Wenn die Texte Umlaute oder ein Esszet enthalten, muss der folgende
%            Befehl bereits an dieser Stelle aktiviert werden:
%            \usepackage[latin1]{inputenc}
% ------------------------------------------------------------------------------
\newcommand{\titel}{Assisted Driving}
\newcommand{\untertitel}{\ }
\newcommand{\art}{SE-Projekt}
\newcommand{\studiengang}{Systems Engineering}

\newcommand{\autor}{\ }

\newcommand{\autorA}{Oleg Tydynyan}
\newcommand{\matrikelnrA}{53 33 13}

\newcommand{\autorB}{Max Wahl}
\newcommand{\matrikelnrB}{53 33 26}

\newcommand{\autorC}{Robert Ledwig}
\newcommand{\matrikelnrC}{53 33 16}

\newcommand{\datum}{\today}
\newcommand{\logo}{HTW_Logo.jpg}


% ben�tigte Packages -----------------------------------------------------------
%   LaTeX-Packages, die ben�tigt werden, sind in die Datei Packages.tex
%   "ausgelagert", um diese Vorlage m�glichst �bersichtlich zu halten.
% ------------------------------------------------------------------------------
% Anpassung des Seitenlayouts --------------------------------------------------
%   siehe Seitenstil.tex
% ------------------------------------------------------------------------------
\usepackage[
    automark, % Kapitelangaben in Kopfzeile automatisch erstellen
    headsepline, % Trennlinie unter Kopfzeile
    ilines % Trennlinie linksb�ndig ausrichten
]{scrpage2}

% Anpassung an Landessprache ---------------------------------------------------
\usepackage[ngerman]{babel}

% Umlaute ----------------------------------------------------------------------
%   Umlaute/Sonderzeichen wie ���� direkt im Quelltext verwenden (CodePage).
%   Erlaubt automatische Trennung von Worten mit Umlauten.
% ------------------------------------------------------------------------------
\usepackage[latin1]{inputenc}
\usepackage[T1]{fontenc}
\usepackage{textcomp} % Euro-Zeichen etc.
\usepackage[right]{eurosym}

% Schrift ----------------------------------------------------------------------
\usepackage{lmodern} % bessere Fonts
\usepackage{relsize} % Schriftgr��e relativ festlegen


% Grafiken ---------------------------------------------------------------------
% Einbinden von JPG-Grafiken erm�glichen
\usepackage[dvips,final]{graphicx}
% hier liegen die Bilder des Dokuments
\graphicspath{{Bilder/}}

% Befehle aus AMSTeX f�r mathematische Symbole z.B. \boldsymbol \mathbb --------
\usepackage{amsmath,amsfonts}

% f�r Index-Ausgabe mit \printindex --------------------------------------------
\usepackage{makeidx}

% Einfache Definition der Zeilenabst�nde und Seitenr�nder etc. -----------------
\usepackage{setspace}
\usepackage{geometry}

\usepackage[printonlyused]{acronym} %Abkürzungsverzeichnis erstellen, Langtext als Fußnote, Auflistung nur bei Verwendung

% zum Umflie�en von Bildern ----------------------------------------------------
\usepackage{floatflt}


% zum Einbinden von Programmcode -----------------------------------------------
\usepackage{listings}
\usepackage{xcolor} 
\definecolor{hellgelb}{rgb}{1,1,0.9}
\definecolor{colKeys}{rgb}{0,0,1}
\definecolor{colIdentifier}{rgb}{0,0,0}
\definecolor{colComments}{rgb}{1,0,0}
\definecolor{colString}{rgb}{0,0.5,0}
\lstset{
    float=hbp,
    basicstyle=\ttfamily\color{black}\small\smaller,
    identifierstyle=\color{colIdentifier},
    keywordstyle=\color{colKeys},
    stringstyle=\color{colString},
    commentstyle=\color{colComments},
    columns=flexible,
    tabsize=2,
    frame=single,
    extendedchars=true,
    showspaces=false,
    showstringspaces=false,
    numbers=left,
    numberstyle=\tiny,
    breaklines=true,
    backgroundcolor=\color{hellgelb},
    breakautoindent=true
}

% URL verlinken, lange URLs umbrechen etc. -------------------------------------
\usepackage{url}

% wichtig f�r korrekte Zitierweise ---------------------------------------------
\usepackage[square]{natbib}

% PDF-Optionen -----------------------------------------------------------------
\usepackage[
    bookmarks,
    bookmarksopen=true,
    colorlinks=true,
% diese Farbdefinitionen zeichnen Links im PDF farblich aus
    linkcolor=black, % einfache interne Verkn�pfungen
    anchorcolor=black,% Ankertext
    citecolor=blue, % Verweise auf Literaturverzeichniseintr�ge im Text
    filecolor=magenta, % Verkn�pfungen, die lokale Dateien �ffnen
    menucolor=black, % Acrobat-Men�punkte
    urlcolor=cyan, 
    backref,
    plainpages=false, % zur korrekten Erstellung der Bookmarks
    pdfpagelabels, % zur korrekten Erstellung der Bookmarks
    hypertexnames=false, % zur korrekten Erstellung der Bookmarks
   %linktocpage % Seitenzahlen anstatt Text im Inhaltsverzeichnis verlinken
]{hyperref}
% Befehle, die Umlaute ausgeben, f�hren zu Fehlern, wenn sie hyperref als Optionen �bergeben werden
\hypersetup{
    pdftitle={\titel \untertitel},
    pdfauthor={\autor},
    pdfcreator={\autor},
    pdfsubject={\titel \untertitel},
    pdfkeywords={\titel \untertitel},
}

% fortlaufendes Durchnummerieren der Fu�noten ----------------------------------
\usepackage{chngcntr}

% f�r lange Tabellen -----------------------------------------------------------
\usepackage{longtable}
\usepackage{array}
\usepackage{ragged2e}
\usepackage{lscape}

% Spaltendefinition rechtsb�ndig mit definierter Breite ------------------------
\newcolumntype{w}[1]{>{\raggedleft\hspace{0pt}}p{#1}}

% Formatierung von Listen �ndern -----------------------------------------------
\usepackage{paralist}

% bei der Definition eigener Befehle ben�tigt
\usepackage{ifthen}

% definiert u.a. die Befehle \todo und \listoftodos
\usepackage{todonotes}

% sorgt daf�r, dass Leerzeichen hinter parameterlosen Makros nicht als Makroendezeichen interpretiert werden
\usepackage{xspace}


% Kopf- und Fu�zeilen, Seitenr�nder etc. ---------------------------------------
% Zeilenabstand 1,5 Zeilen -----------------------------------------------------
\onehalfspacing

% Seitenr�nder -----------------------------------------------------------------
\setlength{\topskip}{\ht\strutbox} % behebt Warnung von geometry
\geometry{paper=a4paper,left=35mm,right=35mm,top=30mm}

% Kopf- und Fu�zeilen ----------------------------------------------------------
\pagestyle{scrheadings}
% Kopf- und Fu�zeile auch auf Kapitelanfangsseiten
\renewcommand*{\chapterpagestyle}{scrheadings} 
% Schriftform der Kopfzeile
\renewcommand{\headfont}{\normalfont}

% Kopfzeile
\ihead{\large{\textsc{\titel}}\\ \small{\untertitel} \\[2ex] \textit{\headmark}}
\chead{}
\ohead{\includegraphics[scale=0.5]{\logo}}
\setlength{\headheight}{21mm} % H�he der Kopfzeile
% Kopfzeile �ber den Text hinaus verbreitern
\setheadwidth[0pt]{textwithmarginpar} 
\setheadsepline[text]{0.4pt} % Trennlinie unter Kopfzeile

% Fu�zeile
\ifoot{\copyright\ \autor}
\cfoot{}
\ofoot{\pagemark}

% sonstige typographische Einstellungen ----------------------------------------

% erzeugt ein wenig mehr Platz hinter einem Punkt
\frenchspacing 

% Schusterjungen und Hurenkinder vermeiden
\clubpenalty = 10000
\widowpenalty = 10000 
\displaywidowpenalty = 10000

% Quellcode-Ausgabe formatieren
\lstset{numbers=left, numberstyle=\tiny, numbersep=5pt, breaklines=true}
\lstset{emph={square}, emphstyle=\color{red}, emph={[2]root,base}, emphstyle={[2]\color{blue}}}

% Fu�noten fortlaufend durchnummerieren
\counterwithout{footnote}{chapter}


% eigene Definitionen f�r Silbentrennung
% Trennvorschl�ge im Text werden mit \" angegeben
% untrennbare W�rter und Ausnahmen von der normalen Trennung k�nnen in dieser
% Datei mittels \hyphenation definiert werden
\hyphenation{Funktions-struktur}


% eigene LaTeX-Befehle
% Eigene Befehle und typographische Auszeichnungen f�r diese

% einfaches Wechseln der Schrift, z.B.: \changefont{cmss}{sbc}{n}
\newcommand{\changefont}[3]{\fontfamily{#1} \fontseries{#2} \fontshape{#3} \selectfont}

% Abk�rzungen mit korrektem Leerraum 
\newcommand{\ua}{\mbox{u.\,a.\ }}
\newcommand{\zB}{\mbox{z.\,B.\ }}
\newcommand{\dahe}{\mbox{d.\,h.\ }}
\newcommand{\Vgl}{Vgl.\ }
\newcommand{\bzw}{bzw.\ }
\newcommand{\evtl}{evtl.\ }

\newcommand{\abbildung}[1]{Abbildung~\ref{fig:#1}}

\newcommand{\bs}{$\backslash$}

% erzeugt ein Listenelement mit fetter �berschrift 
\newcommand{\itemd}[2]{\item{\textbf{#1}}\\{#2}}

% einige Befehle zum Zitieren --------------------------------------------------
\newcommand{\Zitat}[2][\empty]{\ifthenelse{\equal{#1}{\empty}}{\citep{#2}}{\citep[#1]{#2}}}

% zum Ausgeben von Autoren
\newcommand{\AutorName}[1]{\textsc{#1}}
\newcommand{\Autor}[1]{\AutorName{\citeauthor{#1}}}

% verschiedene Befehle um W�rter semantisch auszuzeichnen ----------------------
\newcommand{\NeuerBegriff}[1]{\textbf{#1}}
\newcommand{\Fachbegriff}[1]{\textit{#1}}

\newcommand{\Eingabe}[1]{\texttt{#1}}
\newcommand{\Code}[1]{\texttt{#1}}
\newcommand{\Datei}[1]{\texttt{#1}}

\newcommand{\Datentyp}[1]{\textsf{#1}}
\newcommand{\XMLElement}[1]{\textsf{#1}}
\newcommand{\Webservice}[1]{\textsf{#1}}

% f�r Formelverzeichnis --------------------------------------------------------------------
\DeclareNewTOC[% 
  indent=0pt,% kein Einzug im Verzeichnis 
  hang=2em,% Einzug f�r den Text im Verzeichnis 
  type=equation 
]{loe} 

\AtBeginDocument{% 
  \newcaptionname{ngerman}\equationname{Formel}% 
  \newcaptionname{ngerman}\listequationname{Formelverzeichnis}% 
} 

% f�r Bildquellen ----------------------------------------------------------------------------
\newcommand*{\quelle}{
	\footnotesize Quelle:
}

% Daf�r sorgen, dass Eintr�ge immer dann erzeugt werden k�nnen, wenn auch Tags 
% gesetzt werden: 
\makeatletter 
\newcommand*{\@currententry}{} 
% Zwei amsmath-Anweisungen �ndern: 
\g@addto@macro\make@display@tag{\set@currententry}% 
\def\tagform@#1{\maketag@@@{(\ignorespaces#1\unskip\@@italiccorr)}% 
  \set@currententry} 
\newcommand*{\set@currententry}{% 
  \typeout{set current entry}% 
  \ifx\@currententry\@empty\else 
    \addcontentsline{loe}{equation}{\protect\numberline{\@currentlabel}% 
      \@currententry}% 
    \global\let\@currententry\@empty 
  \fi 
} 

% Neue Benutzeranweisung 
\newcommand*{\equationentry}[1]{% 
  \gdef\@currententry{#1}% 
} 
\makeatother
% ------------------------------------------------------------------------------------------------

% Das eigentliche Dokument -----------------------------------------------------
%   Der eigentliche Inhalt des Dokuments beginnt hier. Die einzelnen Seiten
%   und Kapitel werden in eigene Dateien ausgelagert und hier nur inkludiert.
% ------------------------------------------------------------------------------
\begin{document}

% auch subsubsection nummerieren
\setcounter{secnumdepth}{3}
\setcounter{tocdepth}{3}

% Deckblatt und Abstract ohne Seitenzahl
\ofoot{}
\thispagestyle{plain}
\begin{titlepage}

\begin{center}

\huge{\textbf{\titel}}\\[1.5ex]
\large{\textbf{\untertitel}}\\[4ex]

\LARGE{\textbf{\art}}\\[1.5ex]
\Large{\studiengang}\\[12ex]

\includegraphics[scale=0.9]{htw_logo.jpg}\\[4ex]

\normalsize
\begin{tabular}{w{5.4cm}p{6cm}}\\
\quad \autorA  & \quad \matrikelnrA\\[1.2ex]
\quad \autorB  & \quad \matrikelnrB\\[1.2ex]
\quad \autorC  & \quad \matrikelnrC\\[4ex]
\end{tabular}

\datum

\end{center}

\end{titlepage}

\ofoot{\pagemark}


% -------------------------------------------------------------------------------------
% Verzeichnisse
% -------------------------------------------------------------------------------------

% Seitennummerierung: Vor dem Hauptteil werden die Seiten in gro�en r�mischen Ziffern nummeriert.
\pagenumbering{Roman}
\tableofcontents % Inhaltsverzeichnis

\listoffigures % Abbildungsverzeichnis
\listoftables % Tabellenverzeichnis
\renewcommand{\lstlistlistingname}{Listingverzeichnis}
\lstlistoflistings % Listings-Verzeichnis
\listofequations %Formelverzeichnis

% --------------------------------------------------------------------------------------
% Abk�rzungsverzeichnis
    \addchap{Abk�rzungsverzeichnis}
\begin{acronym}[DDR-SDRAM] % l�ngste Abk�rzung steht in eckigen Klammern
	\setlength{\itemsep}{-\parsep} % geringerer Zeilenabstand
	\input{Inhalt/Abk�rzungen}
        \end{acronym}
% --------------------------------------------------------------------------------------

% arabische Seitenzahlen im Hauptteil ------------------------------------------
\clearpage
\pagenumbering{arabic}

% die Inhaltskapitel werden in "Inhalt.tex" inkludiert -------------------------
% Hier k�nnen die einzelnen Kapitel inkludiert werden. Sie m�ssen in den 
% entsprechenden .TEX-Dateien vorliegen. Die Dateinamen k�nnen nat�rlich 
% angepasst werden.

\chapter{Einleitung}
\label{cha:Einleitung}

% ----------------------------------------------------------------------------------------------------------
% Ziele des Projektes und aktueller Stand
% ----------------------------------------------------------------------------------------------------------
\section{Ziele des Projektes und aktueller Stand}
\chapter{Theoretische Grundlagen}
\label{cha:Theoretische Grundlagen}

Es folgen die theoretischen Grundlagen zu den verwendeten Systemkomponenten aus Kapitel \ref{cha:Systemdesign} Seite \pageref{lab:Umsetzung} f.

% ----------------------------------------------------------------------------------------------------------
% STM32F4 Discovery Evaluation Board
% ----------------------------------------------------------------------------------------------------------
\section{STM32F4 Discovery Evaluation Board}

\begin{equation}
\begin{split}
\text{absoluter Messfehler} = \text{Aufl�sung Sensor} + \text{Aufl�sung FPGA}
\end{split}
\label{eq:absoluter_messfehler}
\equationentry{Errechnung des absoluten Messfehlers}
\end{equation} 
\chapter{Systemdesign}
\label{cha:Systemdesign}

% ----------------------------------------------------------------------------------------------------------
% Umsetzung
% ----------------------------------------------------------------------------------------------------------
\section{Umsetzung}
\label{lab:Umsetzung}

\vspace{1em}
\begin{minipage}{\linewidth}
	\centering
	\includegraphics[width=0.3\linewidth]{Bilder/htw_logo.jpg}
	\captionof{figure}[funktionale Systemarchitektur]{funktionale Systemarchitektur}
	\label{fig:htw_logo}
\end{minipage}

\textbf{Die theoretischen Grundlagen zu den verwendeten Systemkomponenten sowie zum I2C-Bus befinden sich im Kapitel \ref{cha:Systemdesign}.}

\paragraph{Software}
Auf Softwareseite m�ssen sowohl das STM-Board als auch das FPGA-Board programmiert werden. Als Entwicklungsumgebungen werden jeweils Freewarel�sungen verwendet. Das STM-Board wird mit der Entwicklungsumgebung \textit{CooCox CoIDE V1.7.7} in der Programmiersprache C programmiert. Diese Entwicklungsumgebung dient dem Programmieren und Debuggen von ARM Cortex MCU basierten Mikrocontrollerfamilien. Das FPGA-Board wird mit der Software \textit{Altera Quartus II Web Edition V11.0 SP1} in VHDL beschrieben. Die Software f�r den Logic Analyzer hei�t \textit{Saleae Logic 1.1.15}.

\ref{fig:htw_logo}).

sedsadfsadf  \textit{LEs\footnote{\ac{ACC}}} ageafgsfdhsdh



\pagebreak
\paragraph{Materialliste}
Es folgt die verwendete Materialliste f�r das System.

\begin{table}[h]
\centering
\begin{tabular}{llll}
	\hline
	\textbf{Artikel} & \textbf{Anzahl} & \textbf{St�ckpreis [\euro]} & \textbf{Distributor} \\
	\hline
	STM32F4 Disc. Eval. Board & 1 & 15 & Ebay \\
	Altera FPGA Cyclone I inkl. Programmer & 1 & 38 & Ebay \\
	HC-SR04 Ultraschallsensor & 8 & 2 & Ebay \\
	Saleae Logic Analyzer 24Mhz 8CH & 1 & 10 & Ebay \\
	30x Steckbr�cken Buchse-Buchse & 1 & 4 & Ebay \\
	mini-USB 2.0 Kabel A Stecker auf mini B - 1m & 3 & 3 & Ebay \\
	12V 2A Netzteil f. FPGA (Stecker: 2,1/5,5mm) & 1 & 7 & Ebay \\
	CooCox CoIDE V1.7.7 & 1 & - & - \\
	Altera Quartus II Web Edition V11.0 SP1 & 1 & - & Altera \\
	\hline
\end{tabular}
\caption{Materialliste}
\label{tab:materialliste}
\end{table}


\chapter{Systemtest und Optimierung}
\label{cha:Systemtest und Optimierung}

%\newpage
%\lstset{language=C, basicstyle=\footnotesize, showstringspaces=false, tabsize=2}
%\lstinputlisting[label=lst:application,caption= \Code{HC-SR04-Applikation}]{DVD/Listings/application.c}

\chapter{Zusammenfassung und Ausblicke}
\label{cha:Zusammenfassung und Ausblicke}



% Literaturverzeichnis ---------------------------------------------------------
%   Das Literaturverzeichnis wird aus der BibTeX-Datenbank "Bibliographie.bib"
%   erstellt.
% ------------------------------------------------------------------------------
\bibliography{Bibliographie} % Aufruf: bibtex
\bibliographystyle{natdin} % DIN-Stil des Literaturverzeichnisses

% ------------------------------------------------------------------------------
% Anhang
% ------------------------------------------------------------------------------
\begin{appendix}
    \clearpage
    \pagenumbering{roman}
    %Anhang A
    \chapter{Anhang Code}
    \label{sec:Anhang Code}
    % Rand der Aufz�hlungen in Tabellen anpassen
    \setdefaultleftmargin{1em}{}{}{}{}{}
    Die Codedateien befinden sich unter folgenden Pfaden im Anhang auf CD:

\section{Code 1}
\begin{itemize}
\item Pfad 1: \lstinline{\Code\STM\STM32F4_Project_BA\application\hc_sr04_app.h}
\end{itemize}

\section{Code 2}
\begin{itemize}
\item Pfad 2: \lstinline{\Code\STM\STM32F4_Project_BA\application\hc_sr04_app.h}
\end{itemize}
    %Anhang B
    %\chapter{Anhang Bilder}
    %\label{sec:Anhang Bilder}
    % Rand der Aufz�hlungen in Tabellen anpassen
    %\setdefaultleftmargin{1em}{}{}{}{}{}
    %\section{Bilder}

% in dem Ordner \lstinline{VHDL RTL Bilder} auf CD.
\end{appendix}

% Index ------------------------------------------------------------------------
%   Zum Erstellen eines Index, die folgende Zeile auskommentieren.
% ------------------------------------------------------------------------------
%\printindex

\end{document}
