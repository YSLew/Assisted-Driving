% ------------------------------------------------------------------------------
% Formatvorlage f�r Masterarbeiten an der Ohm-Hochschule N�rnberg
% ------------------------------------------------------------------------------
%   erstellt von Stefan Macke, 24.04.2009
%   http://blog.stefan-macke.de

% Dokumentenkopf ---------------------------------------------------------------
%   Diese Vorlage basiert auf "scrreprt" aus dem koma-script.
% ------------------------------------------------------------------------------
\documentclass[
    11pt, % Schriftgr��e
    DIV10,
    ngerman, % f�r Umlaute, Silbentrennung etc.
    a4paper, % Papierformat
    oneside, % einseitiges Dokument
    titlepage, % es wird eine Titelseite verwendet
    parskip=half, % Abstand zwischen Abs�tzen (halbe Zeile)
    headings=normal, % Gr��e der �berschriften verkleinern
    pointlessnumbers, %keine Punkte nach Kapitelnummer
    listof=totoc, % Verzeichnisse im Inhaltsverzeichnis auff�hren
    bibliography=totoc, % Literaturverzeichnis im Inhaltsverzeichnis auff�hren
    index=totoc, % Index im Inhaltsverzeichnis auff��hren
    captions=tableheading, % Beschriftung von Tabellen unterhalb ausgeben
    final % Status des Dokuments (final/draft)
]{scrreprt}

% Meta-Informationen -----------------------------------------------------------
%   Informationen �ber das Dokument, wie z.B. Titel, Autor, Matrikelnr. etc
%   werden in der Datei Meta.tex definiert und k�nnen danach global
%   verwendet werden.
% ------------------------------------------------------------------------------
% Meta-Informationen -----------------------------------------------------------
%   Definition von globalen Parametern, die im gesamten Dokument verwendet
%   werden k�nnen (z.B auf dem Deckblatt etc.).
%
%   ACHTUNG: Wenn die Texte Umlaute oder ein Esszet enthalten, muss der folgende
%            Befehl bereits an dieser Stelle aktiviert werden:
%            \usepackage[latin1]{inputenc}
% ------------------------------------------------------------------------------
\newcommand{\titel}{Assisted Driving}
\newcommand{\untertitel}{\ }
\newcommand{\art}{SE-Projekt}
\newcommand{\studiengang}{Systems Engineering}

\newcommand{\autor}{\ }

\newcommand{\autorA}{Oleg Tydynyan}
\newcommand{\matrikelnrA}{53 33 13}

\newcommand{\autorB}{Max Wahl}
\newcommand{\matrikelnrB}{53 33 26}

\newcommand{\autorC}{Robert Ledwig}
\newcommand{\matrikelnrC}{53 33 16}

\newcommand{\datum}{\today}
\newcommand{\logo}{HTW_Logo.jpg}


% ben�tigte Packages -----------------------------------------------------------
%   LaTeX-Packages, die ben�tigt werden, sind in die Datei Packages.tex
%   "ausgelagert", um diese Vorlage m�glichst �bersichtlich zu halten.
% ------------------------------------------------------------------------------
% Anpassung des Seitenlayouts --------------------------------------------------
%   siehe Seitenstil.tex
% ------------------------------------------------------------------------------
\usepackage[
    automark, % Kapitelangaben in Kopfzeile automatisch erstellen
    headsepline, % Trennlinie unter Kopfzeile
    ilines % Trennlinie linksb�ndig ausrichten
]{scrpage2}

% Anpassung an Landessprache ---------------------------------------------------
\usepackage[ngerman]{babel}

% Umlaute ----------------------------------------------------------------------
%   Umlaute/Sonderzeichen wie ���� direkt im Quelltext verwenden (CodePage).
%   Erlaubt automatische Trennung von Worten mit Umlauten.
% ------------------------------------------------------------------------------
\usepackage[latin1]{inputenc}
\usepackage[T1]{fontenc}
\usepackage{textcomp} % Euro-Zeichen etc.
\usepackage[right]{eurosym}

% Schrift ----------------------------------------------------------------------
\usepackage{lmodern} % bessere Fonts
\usepackage{relsize} % Schriftgr��e relativ festlegen


% Grafiken ---------------------------------------------------------------------
% Einbinden von JPG-Grafiken erm�glichen
\usepackage[dvips,final]{graphicx}
% hier liegen die Bilder des Dokuments
\graphicspath{{Bilder/}}

% Befehle aus AMSTeX f�r mathematische Symbole z.B. \boldsymbol \mathbb --------
\usepackage{amsmath,amsfonts}

% f�r Index-Ausgabe mit \printindex --------------------------------------------
\usepackage{makeidx}

% Einfache Definition der Zeilenabst�nde und Seitenr�nder etc. -----------------
\usepackage{setspace}
\usepackage{geometry}

\usepackage[printonlyused]{acronym} %Abkürzungsverzeichnis erstellen, Langtext als Fußnote, Auflistung nur bei Verwendung

% zum Umflie�en von Bildern ----------------------------------------------------
\usepackage{floatflt}


% zum Einbinden von Programmcode -----------------------------------------------
\usepackage{listings}
\usepackage{xcolor} 
\definecolor{hellgelb}{rgb}{1,1,0.9}
\definecolor{colKeys}{rgb}{0,0,1}
\definecolor{colIdentifier}{rgb}{0,0,0}
\definecolor{colComments}{rgb}{1,0,0}
\definecolor{colString}{rgb}{0,0.5,0}
\lstset{
    float=hbp,
    basicstyle=\ttfamily\color{black}\small\smaller,
    identifierstyle=\color{colIdentifier},
    keywordstyle=\color{colKeys},
    stringstyle=\color{colString},
    commentstyle=\color{colComments},
    columns=flexible,
    tabsize=2,
    frame=single,
    extendedchars=true,
    showspaces=false,
    showstringspaces=false,
    numbers=left,
    numberstyle=\tiny,
    breaklines=true,
    backgroundcolor=\color{hellgelb},
    breakautoindent=true
}

% URL verlinken, lange URLs umbrechen etc. -------------------------------------
\usepackage{url}

% wichtig f�r korrekte Zitierweise ---------------------------------------------
\usepackage[square]{natbib}

% PDF-Optionen -----------------------------------------------------------------
\usepackage[
    bookmarks,
    bookmarksopen=true,
    colorlinks=true,
% diese Farbdefinitionen zeichnen Links im PDF farblich aus
    linkcolor=black, % einfache interne Verkn�pfungen
    anchorcolor=black,% Ankertext
    citecolor=blue, % Verweise auf Literaturverzeichniseintr�ge im Text
    filecolor=magenta, % Verkn�pfungen, die lokale Dateien �ffnen
    menucolor=black, % Acrobat-Men�punkte
    urlcolor=cyan, 
    backref,
    plainpages=false, % zur korrekten Erstellung der Bookmarks
    pdfpagelabels, % zur korrekten Erstellung der Bookmarks
    hypertexnames=false, % zur korrekten Erstellung der Bookmarks
   %linktocpage % Seitenzahlen anstatt Text im Inhaltsverzeichnis verlinken
]{hyperref}
% Befehle, die Umlaute ausgeben, f�hren zu Fehlern, wenn sie hyperref als Optionen �bergeben werden
\hypersetup{
    pdftitle={\titel \untertitel},
    pdfauthor={\autor},
    pdfcreator={\autor},
    pdfsubject={\titel \untertitel},
    pdfkeywords={\titel \untertitel},
}

% fortlaufendes Durchnummerieren der Fu�noten ----------------------------------
\usepackage{chngcntr}

% f�r lange Tabellen -----------------------------------------------------------
\usepackage{longtable}
\usepackage{array}
\usepackage{ragged2e}
\usepackage{lscape}

% Spaltendefinition rechtsb�ndig mit definierter Breite ------------------------
\newcolumntype{w}[1]{>{\raggedleft\hspace{0pt}}p{#1}}

% Formatierung von Listen �ndern -----------------------------------------------
\usepackage{paralist}

% bei der Definition eigener Befehle ben�tigt
\usepackage{ifthen}

% definiert u.a. die Befehle \todo und \listoftodos
\usepackage{todonotes}

% sorgt daf�r, dass Leerzeichen hinter parameterlosen Makros nicht als Makroendezeichen interpretiert werden
\usepackage{xspace}


% Kopf- und Fu�zeilen, Seitenr�nder etc. ---------------------------------------
\input{Seitenstil}

% eigene Definitionen f�r Silbentrennung
\include{Silbentrennung}

% eigene LaTeX-Befehle
% Eigene Befehle und typographische Auszeichnungen f�r diese

% einfaches Wechseln der Schrift, z.B.: \changefont{cmss}{sbc}{n}
\newcommand{\changefont}[3]{\fontfamily{#1} \fontseries{#2} \fontshape{#3} \selectfont}

% Abk�rzungen mit korrektem Leerraum 
\newcommand{\ua}{\mbox{u.\,a.\ }}
\newcommand{\zB}{\mbox{z.\,B.\ }}
\newcommand{\dahe}{\mbox{d.\,h.\ }}
\newcommand{\Vgl}{Vgl.\ }
\newcommand{\bzw}{bzw.\ }
\newcommand{\evtl}{evtl.\ }

\newcommand{\abbildung}[1]{Abbildung~\ref{fig:#1}}

\newcommand{\bs}{$\backslash$}

% erzeugt ein Listenelement mit fetter �berschrift 
\newcommand{\itemd}[2]{\item{\textbf{#1}}\\{#2}}

% einige Befehle zum Zitieren --------------------------------------------------
\newcommand{\Zitat}[2][\empty]{\ifthenelse{\equal{#1}{\empty}}{\citep{#2}}{\citep[#1]{#2}}}

% zum Ausgeben von Autoren
\newcommand{\AutorName}[1]{\textsc{#1}}
\newcommand{\Autor}[1]{\AutorName{\citeauthor{#1}}}

% verschiedene Befehle um W�rter semantisch auszuzeichnen ----------------------
\newcommand{\NeuerBegriff}[1]{\textbf{#1}}
\newcommand{\Fachbegriff}[1]{\textit{#1}}

\newcommand{\Eingabe}[1]{\texttt{#1}}
\newcommand{\Code}[1]{\texttt{#1}}
\newcommand{\Datei}[1]{\texttt{#1}}

\newcommand{\Datentyp}[1]{\textsf{#1}}
\newcommand{\XMLElement}[1]{\textsf{#1}}
\newcommand{\Webservice}[1]{\textsf{#1}}

% f�r Formelverzeichnis --------------------------------------------------------------------
\DeclareNewTOC[% 
  indent=0pt,% kein Einzug im Verzeichnis 
  hang=2em,% Einzug f�r den Text im Verzeichnis 
  type=equation 
]{loe} 

\AtBeginDocument{% 
  \newcaptionname{ngerman}\equationname{Formel}% 
  \newcaptionname{ngerman}\listequationname{Formelverzeichnis}% 
} 

% f�r Bildquellen ----------------------------------------------------------------------------
\newcommand*{\quelle}{
	\footnotesize Quelle:
}

% Daf�r sorgen, dass Eintr�ge immer dann erzeugt werden k�nnen, wenn auch Tags 
% gesetzt werden: 
\makeatletter 
\newcommand*{\@currententry}{} 
% Zwei amsmath-Anweisungen �ndern: 
\g@addto@macro\make@display@tag{\set@currententry}% 
\def\tagform@#1{\maketag@@@{(\ignorespaces#1\unskip\@@italiccorr)}% 
  \set@currententry} 
\newcommand*{\set@currententry}{% 
  \typeout{set current entry}% 
  \ifx\@currententry\@empty\else 
    \addcontentsline{loe}{equation}{\protect\numberline{\@currentlabel}% 
      \@currententry}% 
    \global\let\@currententry\@empty 
  \fi 
} 

% Neue Benutzeranweisung 
\newcommand*{\equationentry}[1]{% 
  \gdef\@currententry{#1}% 
} 
\makeatother
% ------------------------------------------------------------------------------------------------

% Das eigentliche Dokument -----------------------------------------------------
%   Der eigentliche Inhalt des Dokuments beginnt hier. Die einzelnen Seiten
%   und Kapitel werden in eigene Dateien ausgelagert und hier nur inkludiert.
% ------------------------------------------------------------------------------
\begin{document}

% auch subsubsection nummerieren
\setcounter{secnumdepth}{3}
\setcounter{tocdepth}{3}

% Deckblatt und Abstract ohne Seitenzahl
\ofoot{}
\thispagestyle{plain}
\begin{titlepage}

\begin{center}

\huge{\textbf{\titel}}\\[1.5ex]
\large{\textbf{\untertitel}}\\[4ex]

\LARGE{\textbf{\art}}\\[1.5ex]
\Large{\studiengang}\\[12ex]

\includegraphics[scale=0.9]{htw_logo.jpg}\\[4ex]

\normalsize
\begin{tabular}{w{5.4cm}p{6cm}}\\
\quad \autorA  & \quad \matrikelnrA\\[1.2ex]
\quad \autorB  & \quad \matrikelnrB\\[1.2ex]
\quad \autorC  & \quad \matrikelnrC\\[4ex]
\end{tabular}

\datum

\end{center}

\end{titlepage}

% ----------------------------------------------------------------------------------------------------------
% Stichworte
% ----------------------------------------------------------------------------------------------------------
\section*{Stichworte}
\label{sec:Stichworte}
%Field-Programmable Gate Array, Mikrocontroller, Ultraschallsensorik, Pulsweitenmodulation, Inter-Integrated Circuit


% ----------------------------------------------------------------------------------------------------------
% Zusammenfassung
% ----------------------------------------------------------------------------------------------------------
\section*{Zusammenfassung}
\label{sec:Zusammenfassung}
%Bei \textit{Berner\&Mattner Systemtechnik GmbH} wird ein Modellfahrzeug zu Ausbildungs- und Forschungszwecken im Bereich der Antriebs- und Fahrerassistenzfunktionen aufgebaut. Dieses %Modellfahrzeug soll zun�chst �ber eine Fernsteuerung steuerbar gemacht werden und sp�ter �ber \textit{ADAS\footnote{\ac{ADAS}}}-Funktionen erweitert werden, sodass beispielsweise eine %autonome Fahrweise �ber eine Spurerkennung m�glich wird.


% ----------------------------------------------------------------------------------------------------------
% Keywords
% ----------------------------------------------------------------------------------------------------------
\section*{Keywords}
\label{sec:Keywords}

Englisch.

% ----------------------------------------------------------------------------------------------------------
% Abstract
% ----------------------------------------------------------------------------------------------------------
\section*{Abstract}
\label{sec:Abstract}

Englisch.
\ofoot{\pagemark}


% -------------------------------------------------------------------------------------
% Verzeichnisse
% -------------------------------------------------------------------------------------

% Seitennummerierung: Vor dem Hauptteil werden die Seiten in gro�en r�mischen Ziffern nummeriert.
\pagenumbering{Roman}
\tableofcontents % Inhaltsverzeichnis

\listoffigures % Abbildungsverzeichnis
\listoftables % Tabellenverzeichnis
\renewcommand{\lstlistlistingname}{Listingverzeichnis}
\lstlistoflistings % Listings-Verzeichnis
\listofequations %Formelverzeichnis

% --------------------------------------------------------------------------------------
% Abk�rzungsverzeichnis
    \addchap{Abk�rzungsverzeichnis}
	\begin{acronym}[DDR-SDRAM] % l�ngste Abk�rzung steht in eckigen Klammern
		\setlength{\itemsep}{-\parsep} % geringerer Zeilenabstand
		\input{Inhalt/Abk�rzungen}
        	\end{acronym}
% --------------------------------------------------------------------------------------

% arabische Seitenzahlen im Hauptteil ------------------------------------------
\clearpage
\pagenumbering{arabic}

% die Inhaltskapitel werden in "Inhalt.tex" inkludiert -------------------------
% Hier k�nnen die einzelnen Kapitel inkludiert werden. Sie m�ssen in den 
% entsprechenden .TEX-Dateien vorliegen. Die Dateinamen k�nnen nat�rlich 
% angepasst werden.

\chapter{Einleitung}
\label{cha:Einleitung}

% ----------------------------------------------------------------------------------------------------------
% Ziele der Arbeit und aktueller Stand
% ----------------------------------------------------------------------------------------------------------
\section{Ziele der Arbeit und aktueller Stand}


% ----------------------------------------------------------------------------------------------------------
% Gliederung der Arbeit
% ----------------------------------------------------------------------------------------------------------
\section{Gliederung der Arbeit}

\chapter{Lastenheft}
\label{cha:Lastenheft}

% ----------------------------------------------------------------------------------------------------------
% Section
% ----------------------------------------------------------------------------------------------------------
\section{sgsgsdfgsdfg}
\chapter{Pflichtenheft}
\label{cha:Pflichtenheft}

% ----------------------------------------------------------------------------------------------------------
% Section
% ----------------------------------------------------------------------------------------------------------
\section{Projektziel/Einf�hrung}

Ziel des Projekts ist es ein System zu entwickeln, welches in der Lage ist (Vorfahrts-) Schilder im Stra�enverkehr autonom zu erfassen. Die Erfassung erfolgt ? w�hrend der Fahrt ? mit Hilfe einer Kamera. Ist das System aktiv, l�uft die Kamera stets mit und es werden alle erfassten Bilder in die zentrale Recheneinheit (CPU) weitergeleitet. Dort werden diese mittels einer Bilderfassungssoftware analysiert und die Informationen der erkannten Verkehrsschilder an den Fahrer weitergeleitet. Die Ausgabe erfolgt visuell; auf einem Display wird das erfasste (Vorfahrts-) Schild durch ein Icon dargestellt. 

\vspace{1em}
\begin{minipage}{\linewidth}
	\centering
	\includegraphics[width=0.7\linewidth]{Bilder/Einfuehrung_Bild.png}
	\captionof{figure}[Modell]{Modell}
	\label{fig:Einfuehrung_Bild}
\end{minipage}

\section{Anforderungsanalyse}

\subsection{Muss-Anforderungen}

Die hier beschriebenen Punkte m�ssen erf�llt werden, um das Projekt erfolgreich abzuschlie�en.

\begin{enumerate}
\item Es werden mindestens zwei deutsche Vorfahrtsschilder (Stopp-Schild und ?Vorfahrt beachten?) korrekt erkannt.
\item 51\% aller Schilder werden korrekt erkannt. 
\item Die Bilderfassung erfolgt korrekt (siehe M2) bis zu einer Fahrzeuggeschwindigkeit von 30 km/h. 
\item Die Bilderfassung erfolgt mit einer Abtastrate von mindestens 1 Hz (max. m�glicher Schilder-Abstand dadurch ca. 10m bei 30 km/h).
\item Die von der Kamera aufgenommenen Bilder werden in eine niedrigere Aufl�sung umgerechnet. Diese und die maximal m�gliche Aufl�sung der aufgenommenen Bilder sind zu bestimmen (unter Beachtung von M2 und M4).
\item Die Ausgabe des erkannten Zeichens erfolgt �ber eine LED (eine je erkanntes Schild) nach maximal 1 Sekunde.
\end{enumerate}

\subsection{Soll-Anforderungen}

Die folgenden Anforderungen sollen umgesetzt werden. Sie sind fest in den Projektplan integriert, sind jedoch nicht projektentscheidend. 

\begin{enumerate}
\item Es werden alle deutschen Vorfahrtszeichen korrekt erkannt.
\item Die Bilderfassung erfolgt korrekt (siehe M2) bis zu einer Fahrzeuggeschwindigkeit von 60 km/h. 
\item Die Bilderfassung erfolgt mit einer Abtastrate von mindestens 2 Hz (max. m�glicher Schilder-Abstand dadurch ca. 10m bei 60 km/h).
\item Die Geschwindigkeit des Fahrzeuges, in dem das System verbaut wird, soll �ber ein CAN-Bus-System erfasst werden. Bei der �berschreitung von 60 km/h soll eine Warnung ausgegeben werden. 
\item Die Anzeige des erkannten Zeichens soll �ber ein Display erfolgen, in welchem das erkannte Verkehrszeichen abgebildet wird (nach max. 0,5 Sekunden), das Icon erscheint 15 s oder bis zur Erkennung eines neuen Verkehrszeichens.
\end{enumerate}

\subsection{Kann-Anforderungen}

Die folgenden Anforderungen sind optional und k�nnen im Laufe des Projektes erreicht werden, bzw. dienen als Ausgangspunkt f�r m�gliche Erweiterungen. Sie sind nicht notwendig, um das Projekt zu einem erfolgreichen Abschluss zu f�hren.

\begin{enumerate}
\item Das System kann weitere  Verkehrsschilder erkennen. 
\item Die Bilderfassung wird mit einem Smartphone durchgef�hrt, welches vom Nutzer gestellt wird.
\item Die erkannten Verkehrszeichen werden �ber ein HUD (Head-up-Display) in die Windschutzscheibe in den Sichtbereich des Fahrers projiziert. 
\item Zus�tzlich zur Anzeige wird ein akustischer Warnton ausgel�st, wenn z.B. ein Stoppschild ohne Anhalten �berfahren wurde.
\item Durch Erweiterung auf eine Car-2-Car Communication (WLAN- und GPS-basiert) wird vor einem bevorstehenden Unfall gewarnt, andere Verkehrsteilnehmer in der N�he werden z.B. bei �berfahrenen Stopp-Schildern gewarnt und ggf. automatisch gebremst.
\end{enumerate}

\section{Entwicklungstools}

Die zu entwerfende Software wird mit Hilfe der Entwicklungsumgebung ''{}Visual Studio 2013''{} \footnote{\url{https://www.visualstudio.com/} \\letzter Zugriff: 25.10.2015} entworfen, unter Nutzung der Open-Source-Bibliothek ''{}OpenCV''{} \footnote{\url{http://opencv.org/} \\letzter Zugriff: 25.10.2015}. Das fertige System soll auf einer f�r diese Anwendung optimierten und mit dem Build-Werkzeug ''{}yocto''{} \footnote{\url{https://www.yoctoproject.org/} \\letzter Zugriff: 25.10.2015} erstellten Linux-Distribution in Betrieb genommen werden. F�r deren Start- und Flashvorgang auf der gew�hlten Hardware kommt der Open-Source-Bootloader ''{}U-Boot''{} \footnote{\url{http://www.denx.de/wiki/U-Boot/WebHome} \\letzter Zugriff: 25.10.2015} zum Einsatz. 

\section{Hauptkomponenten}

\subsection{Algorithmischer Ansatz}

Die von der Kamera erfassten Bilder werden unter Verwendung geeigneter Filter-Algorithmen vorbearbeitet und skaliert. Anschlie�end durchlaufen die nun erhaltenen Daten einen f�r die einzelnen Verkehrszeichen spezifischen Farbfilter. Daraufhin k�nnen geeignete Objekterkennende Algorithmen (Feature-Tracking und ggf. Viola-Jones) genutzt werden, um die gew�nschten Schilder eindeutig zu identifizieren. Die genutzten Algorithmen werden so gew�hlt, dass auch nur teilweise erkennbare sowie farblich variierende Zeichen (z.B. bei unterschiedlichen Lichtstimmungen) erkannt werden k�nnen und dabei gleichzeitig ausreichend Performance garantieren.

\subsection{Hardware}

Die fertig entwickelten Programme werden auf ein ?MarS-Board? der Firma Embest Technologies �berspielt. Dieses stellt neben einem ausreichend schnellen ARM Cortex-A9-Prozessor und zwei USB-Schnittstellen an denen direkt die ben�tigte Kamera angebunden werden kann, auch einen speziellen Grafikprozessor (Vivante GC2000) der zur Vorverarbeitung der eingehende Bilder genutzt werden kann. Eine Ethernet-Schnittstelle erlaubt ebenfalls den Zugriff auf Kameras �ber ein Netzwerk.

\subsection{Kamera}

Verwendet wird die USB3.0-Kamera mxBlueFox3 der Firma Matrix Vision. Diese bietet mit 10 Megapixel ein Vielfaches der geforderten Aufl�sung und steht im Labor der HTW zur Verf�gung.


\section{Probleme}

\subsection{Potentielle Probleme im Entwicklungsprozess}

Gr��te Herausforderung im Entwicklungsprozess wird die Anpassung der gew�hlten Algorithmen sein, so dass diese einen akzeptablen Kompromiss zwischen Performance (maximale Bild�bertragungsrate bei gew�nschter Aufl�sung) und Erfolgsrate der Objekterkennung bietet. Es ist zu untersuchen unter welchen Bedingungen das System noch einwandfrei arbeitet und welche Verkehrszeichen besonders schwierig zu erkennen sind. Die folgende Optimierung des Codes wird aufgrund der Komplexit�t der Bildverarbeitenden Algorithmen viel Zeit in Anspruch nehmen.

\subsection{Externe Einfl�sse im Betrieb}

Das finale System wird stets nur eine begrenzte Erfolgsquote bei der Erkennung der Verkehrszeichen aufweisen. Im Stra�enverkehr k�nnen Schilder verdreht oder verdreckt sein und je nach Tageszeit und Wetter unterschiedlich schwer wahr zu nehmen sein. Ziel muss es sein, diese St�reinfl�sse zu identifizieren und klare Regeln zu definieren, unter denen das System noch einwandfrei arbeitet.


\section{Ablaufplan}

\vspace{1em}
\begin{minipage}{\linewidth}
	\centering
	\includegraphics[width=1\linewidth]{Bilder/Ablaufplan.png}
	\captionof{figure}[Ablaufplan]{Ablaufplan}
	\label{fig:Ablaufplan}
\end{minipage}


\section{Phasenplan}

\begin{table}[h]
\centering
\begin{tabular}{lll}
	\hline
	\textbf{Phase} & \textbf{Ziel/Meilenstein} & \textbf{Zeitraum} \\
	\hline
	Recherche & Lastenheft & 25.04.15 - 04.05.15 \\
	Konzeptphase & Pflichtenheft & 28.04.15 - 18.06.15 \\
	Entwurfsphase & Fertigstellung Algorithmen & 05.06.15 - 03.07.15 \\
	Fertigungsphase & Lauff�higes System & 25.06.15 - 11.12.15 \\
	Betriebsphase & Abgeschlossenes Projekt & 13.11.15 - 07.01.16 \\
	Pr�sentationsphase & Abgabe Projekt & 21.05.15 - 11.02.16 \\
	\hline
\end{tabular}
\caption{Phasenplan}
\label{tab:Phasenplan}
\end{table}

Der Phasenplan des Projektes besteht insgesamt aus sieben Phasen. Diese Phasen besitzen teils einen flie�end zeitlichen �bergang zueinander. 

Zuerst erfolgt in der Recherchephase (April bis Mai 2015) das Vertraut-Machen der Projektmitglieder mit der Entwicklungsumgebung "Microsoft Visual Studio" in Kombination mit den OpenCV Bibliotheken. Au�erdem werden die Anforderungen (Spezifikationen) an das Projekt definiert; das Lastenheft entsteht. 

Zeitlich versetzt zur Recherchephase erfolgt bereits die Konzeptphase (April bis Juni 2015), die in dem Pflichtenheft endet. Die Projektanforderungen aus dem Lastenheft werden validiert und es erfolgt eine Aufschl�sselung nach Muss-, Soll-, und Kann-Kriterien hinsichtlich Umsetzbarkeit. Ebenso wird nach diesen Kriterien die Hard- und Software abgeleitet sowie ein erster algorithmischer Ansatz zur Probleml�sung erstellt.

Von Juni bis Juli 2015 verl�uft die Entwurfsphase. Hierin informieren sich die Projektteilnehmer �ber die Algorithmik zur Objekterkennung. Erste Testmuster zur Detektierung bestimmter Bildausschnitte werden erstellt. Diese Vorabuntersuchung erfolgt zun�chst unabh�ngig von der sp�ter zu verwendenden Hardware.

In der Ende Juni 2015 startenden Fertigungsphase erfolgt zun�chst die Beschaffung der Hardware, deren anschlie�ende Inbetriebnahme (Schaffung Softwarebasis Mars-Board) sowie die Code-Implementierung. Ist die Fertigungsphase abgeschlossen, ist der Meilenstein "{}lauff�higes System"{} erreicht (Dezember 2015).

Die im November 2015 beginnende Betriebsphase umfasst die Implementierung des Gesamtsystems in der Fahrzeugumgebung. Praktische Tests werden durchgef�hrt sowie eventuelle softwaretechnische Fehler behoben. Im Januar 2016 ist der Abschluss der Betriebsphase geplant.

Die Pr�sentationsphase erfolgt parallel zu allen anderen erw�hnten Phasen. Es erfolgen hierbei die Pr�sentationen zu den Zwischenst�nden, finalen Abschl�ssen des Projektes und das Pflegen der Dokumentation. Die Pr�sentationsphase endet im Februar 2016 mit der Projektabgabe.

\chapter{Systemarchitektur}
\label{cha:Systemarchitektur}

% ----------------------------------------------------------------------------------------------------------
% Umsetzung
% ----------------------------------------------------------------------------------------------------------
\section{funktionale Systemarchitektur}
\section{Ableitung der Hardwareanforderungen}
\section{Ableitung der Softwareanforderungen}

\section{Architektur des Gesamtsystems}

\subsection{Technische Systemarchitektur}

\subsubsection{statisch}
\subsubsection{dynamisch}

\chapter{Softwarearchitektur}
\label{cha:Softwarearchitektur}

\section{Theoretische Grundlagen der verwendeten Algorithmik}

ghjgjhgjhg\Zitat{formen}dfgsdfgsdf\Zitat{objekt}dfgsdfgsdg\Zitat{doku}dfgdgsgsdfg

\section{Softwarearchitektur}


\chapter{Systemtest und Optimierung}
\label{cha:Systemtest und Optimierung}

%\newpage
%\lstset{language=C, basicstyle=\footnotesize, showstringspaces=false, tabsize=2}
%\lstinputlisting[label=lst:application,caption= \Code{HC-SR04-Applikation}]{DVD/Listings/application.c}

\chapter{Zusammenfassung und Ausblicke}
\label{cha:Zusammenfassung und Ausblicke}



% Literaturverzeichnis ---------------------------------------------------------
%   Das Literaturverzeichnis wird aus der BibTeX-Datenbank "Bibliographie.bib"
%   erstellt.
% ------------------------------------------------------------------------------
\bibliography{Bibliographie} % Aufruf: bibtex
\bibliographystyle{natdin} % DIN-Stil des Literaturverzeichnisses

%\include{Erklaerung} % Selbst�ndigkeitserkl�rung

% ------------------------------------------------------------------------------
% Anhang
% ------------------------------------------------------------------------------
\begin{appendix}
    \clearpage
    \pagenumbering{roman}

    %Anhang A
    \chapter{Anhang Code}
    \label{sec:Anhang Code}
    % Rand der Aufz�hlungen in Tabellen anpassen
    \setdefaultleftmargin{1em}{}{}{}{}{}
    Die Codedateien befinden sich unter folgenden Pfaden im Anhang auf CD:

\section{C-Code}
%\begin{itemize}
%\item HC-SR04-Applikation (Header File): \lstinline{\Code\STM\STM32F4_Project_BA\application\hc_sr04_app.h}
%\item HC-SR04-Applikation (Source File): \lstinline{\Code\STM\STM32F4_Project_BA\application\src\hc_sr04_app.c}
%\end{itemize}

    %Anhang B
    \chapter{Anhang Bilder}
    \label{sec:Anhang Bilder}
    \setdefaultleftmargin{1em}{}{}{}{}{}
    \section{Bilder}

% in dem Ordner \lstinline{VHDL RTL Bilder} auf CD.
\end{appendix}

% Index ------------------------------------------------------------------------
%   Zum Erstellen eines Index, die folgende Zeile auskommentieren.
% ------------------------------------------------------------------------------
%\printindex

\end{document}
